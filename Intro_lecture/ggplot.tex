% Intro slides for WK-5 ``Data Visualization Using R''
% Ecological Society of America meeting
% August 2013
%
% Original slides by: Karthik Ram, karthik.ram@gmail.com
% Modified by: Naupaka Zimmerman, naupaka@gmail.com
% Licence, CC-BY

\documentclass{beamer}\usepackage[]{graphicx}\usepackage[]{color}
%% maxwidth is the original width if it is less than linewidth
%% otherwise use linewidth (to make sure the graphics do not exceed the margin)
\makeatletter
\def\maxwidth{ %
  \ifdim\Gin@nat@width>\linewidth
    \linewidth
  \else
    \Gin@nat@width
  \fi
}
\makeatother

\definecolor{fgcolor}{rgb}{0.345, 0.345, 0.345}
\newcommand{\hlnum}[1]{\textcolor[rgb]{0.686,0.059,0.569}{#1}}%
\newcommand{\hlstr}[1]{\textcolor[rgb]{0.192,0.494,0.8}{#1}}%
\newcommand{\hlcom}[1]{\textcolor[rgb]{0.678,0.584,0.686}{\textit{#1}}}%
\newcommand{\hlopt}[1]{\textcolor[rgb]{0,0,0}{#1}}%
\newcommand{\hlstd}[1]{\textcolor[rgb]{0.345,0.345,0.345}{#1}}%
\newcommand{\hlkwa}[1]{\textcolor[rgb]{0.161,0.373,0.58}{\textbf{#1}}}%
\newcommand{\hlkwb}[1]{\textcolor[rgb]{0.69,0.353,0.396}{#1}}%
\newcommand{\hlkwc}[1]{\textcolor[rgb]{0.333,0.667,0.333}{#1}}%
\newcommand{\hlkwd}[1]{\textcolor[rgb]{0.737,0.353,0.396}{\textbf{#1}}}%

\usepackage{framed}
\makeatletter
\newenvironment{kframe}{%
 \def\at@end@of@kframe{}%
 \ifinner\ifhmode%
  \def\at@end@of@kframe{\end{minipage}}%
  \begin{minipage}{\columnwidth}%
 \fi\fi%
 \def\FrameCommand##1{\hskip\@totalleftmargin \hskip-\fboxsep
 \colorbox{shadecolor}{##1}\hskip-\fboxsep
     % There is no \\@totalrightmargin, so:
     \hskip-\linewidth \hskip-\@totalleftmargin \hskip\columnwidth}%
 \MakeFramed {\advance\hsize-\width
   \@totalleftmargin\z@ \linewidth\hsize
   \@setminipage}}%
 {\par\unskip\endMakeFramed%
 \at@end@of@kframe}
\makeatother

\definecolor{shadecolor}{rgb}{.97, .97, .97}
\definecolor{messagecolor}{rgb}{0, 0, 0}
\definecolor{warningcolor}{rgb}{1, 0, 1}
\definecolor{errorcolor}{rgb}{1, 0, 0}
\newenvironment{knitrout}{}{} % an empty environment to be redefined in TeX

\usepackage{alltt}
\usepackage{listings}
\usepackage{inconsolata}
\setbeamertemplate{frametitle}[default][center]
\usepackage{url}
\usepackage{color}
\setcounter{secnumdepth}{-1}
\usetheme{default}
\defbeamertemplate*{title page}{customized}[1][]
{
  \begin{center}
  \usebeamerfont{title}\inserttitle\par
  \usebeamerfont{subtitle}\usebeamercolor[fg]{subtitle}\insertsubtitle\par
  \bigskip
  \bigskip
  \bigskip
  \usebeamerfont{author}\insertauthor\par
  \usebeamerfont{date}\insertdate\par
  \end{center}
}
\addtobeamertemplate{frametitle}{\vspace*{0.5cm}}{}

% --------------------------------------------------------------
% --------------------------------------------------------------
% --------------------------------------------------------------

% Setting up some knitr options



% --------------------------------------------------------------
% --------------------------------------------------------------
% --------------------------------------------------------------
\IfFileExists{upquote.sty}{\usepackage{upquote}}{}

\begin{document}
% \SweaveOpts{concordance=TRUE} % deprecated


\title{Data Visualization Using R \& ggplot2}
\author{Naupaka Zimmerman \and Karthik Ram \and Andrew Tredennick}
\date{August 4, 2013}
\maketitle

% --------------------------------------------------------------

\begin{frame}[fragile]
\frametitle{Some housekeeping}
Install some packages
\begin{knitrout}\footnotesize
\definecolor{shadecolor}{rgb}{0.969, 0.969, 0.969}\color{fgcolor}\begin{kframe}
\begin{alltt}
\hlkwd{install.packages}(\hlstr{"ggplot2"}, dependencies = TRUE)
\hlkwd{install.packages}(\hlstr{"plyr"})
\hlkwd{install.packages}(\hlstr{"ggthemes"})
\hlkwd{install.packages}(\hlstr{"reshape2"})
\hlkwd{install.packages}(\hlstr{"gridExtra"})
\hlkwd{install.packages}(\hlstr{"devtools"})
\hlcom{# Then a few packages to acquire data from the web to visualize}
\hlkwd{install.packages}(\hlstr{"rfisheries"})
\hlkwd{install.packages}(\hlstr{"rgbif"})
\hlkwd{install.packages}(\hlstr{"taxize"})
\hlcom{# optional}
\hlkwd{install_github}(\hlstr{"rWBclimate"}, \hlstr{"ropensci"})
\end{alltt}
\end{kframe}
\end{knitrout}

\end{frame}

% --------------------------------------------------------------
% --------------------------------------------------------------

\section*{Why \texttt{ggplot2}?}
\frame{\sectionpage}

% --------------------------------------------------------------
% --------------------------------------------------------------

\begin{frame}[fragile]
\frametitle{Why \texttt{ggplot2}?}
\begin{itemize}
\item More elegant \& compact code than with base graphics\\
\item More aethetically pleasing defaults than lattice\\
\item Very powerful for exploratory data analysis\\
\end{itemize}
\end{frame}

% --------------------------------------------------------------

\begin{frame}[fragile]
\frametitle{Why \texttt{ggplot2}?}
\begin{itemize}
\item `gg' is for `grammar of graphics' (term by Lee Wilkinson)\\
\item A set of terms that defines the basic components of a plot\\
\item Used to produce figures using coherant, consistant syntax\\
\end{itemize}
\end{frame}

% --------------------------------------------------------------

\begin{frame}[fragile]
\frametitle{Why \texttt{ggplot2}?}
\begin{itemize}
\item Supports a continuum of expertise:
\item Easy to get started, plenty of power for complex figures 
\end{itemize}
\end{frame}

% --------------------------------------------------------------
% --------------------------------------------------------------

\section*{The Grammar}
\frame{\sectionpage}

% --------------------------------------------------------------
% --------------------------------------------------------------

\begin{frame}[fragile]
\frametitle{Some terminology}
\begin{columns}[t]

\begin{column}[T]{3cm}
\begin{itemize}
    \item \textbf{data}
\end{itemize}
\end{column}

\begin{column}[T]{8cm}
\begin{itemize}
    \item Must be a data.frame
    \item Gets pulled into the ggplot() object
\end{itemize}
\end{column}

\end{columns}
\end{frame}

% --------------------------------------------------------------

\begin{frame}[fragile]
\frametitle{The iris dataset}
\begin{knitrout}\footnotesize
\definecolor{shadecolor}{rgb}{0.969, 0.969, 0.969}\color{fgcolor}\begin{kframe}
\begin{alltt}
\hlkwd{head}(iris)
\end{alltt}
\begin{verbatim}
##   Sepal.Length Sepal.Width Petal.Length Petal.Width Species
## 1          5.1         3.5          1.4         0.2  setosa
## 2          4.9         3.0          1.4         0.2  setosa
## 3          4.7         3.2          1.3         0.2  setosa
## 4          4.6         3.1          1.5         0.2  setosa
## 5          5.0         3.6          1.4         0.2  setosa
## 6          5.4         3.9          1.7         0.4  setosa
\end{verbatim}
\end{kframe}
\end{knitrout}

\end{frame}

% --------------------------------------------------------------
% --------------------------------------------------------------

\section*{Aesthetics}
\frame{\sectionpage}

% --------------------------------------------------------------
% --------------------------------------------------------------


\begin{frame}[fragile]
\frametitle{Some terminology}
\begin{columns}[t]

\begin{column}[T]{3cm}
\begin{itemize}
    \item \textbf{\color{gray}data}
    \item \textbf{aes}thetics
\end{itemize}
\end{column}

\begin{column}[T]{8cm}
\begin{itemize}
    \item \textbf{How your data is represented visually}
        \begin{itemize}
        \item \emph{a.k.a. mapping}
        \end{itemize}
    \item which data on the x
    \item which data on the y
    \item but also: {\color{red}color}, {\LARGE{size}}, shape, transparency
\end{itemize}
\end{column}

\end{columns}
\end{frame}

% --------------------------------------------------------------

\begin{frame}[fragile]
\frametitle{Let's try an example}
\begin{knitrout}\footnotesize
\definecolor{shadecolor}{rgb}{0.969, 0.969, 0.969}\color{fgcolor}\begin{kframe}
\begin{alltt}
myplot <- \hlkwd{ggplot}(data = iris, \hlkwd{aes}(x = Sepal.Length, y = Sepal.Width))
\hlkwd{summary}(myplot)
\end{alltt}
\begin{verbatim}
## data: Sepal.Length, Sepal.Width, Petal.Length,
##   Petal.Width, Species [150x5]
## mapping:  x = Sepal.Length, y = Sepal.Width
## faceting: facet_null()
\end{verbatim}
\end{kframe}
\end{knitrout}

\end{frame}

% --------------------------------------------------------------
% --------------------------------------------------------------

\section*{Geoms}
\frame{\sectionpage}

% --------------------------------------------------------------
% --------------------------------------------------------------

\begin{frame}[fragile]
\frametitle{Some terminology}
\begin{columns}[t]

\begin{column}[T]{3cm}
\begin{itemize}
    \item \textbf{\color{gray}data}
    \item \textbf{\color{gray}aesthetics}
    \item \textbf{geom}etry
\end{itemize}
\end{column}

\begin{column}[T]{8cm}
\begin{itemize}
    \item \textbf{The geometric objects in the plot}
    \item points, lines, polygons, etc
    \item shortcut functions: geom\_point(), geom\_bar(), geom\_line()
\end{itemize}
\end{column}

\end{columns}
\end{frame}

% --------------------------------------------------------------

\begin{frame}[fragile]
\frametitle{Basic structure}
\begin{knitrout}\footnotesize
\definecolor{shadecolor}{rgb}{0.969, 0.969, 0.969}\color{fgcolor}\begin{kframe}
\begin{alltt}
\hlkwd{ggplot}(data = iris, \hlkwd{aes}(x = Sepal.Length, y = Sepal.Width))
 + \hlkwd{geom_point}()

myplot <- \hlkwd{ggplot}(data = iris, \hlkwd{aes}(x = Sepal.Length, y = Sepal.Width))
myplot + \hlkwd{geom_point}()
\end{alltt}
\end{kframe}
\end{knitrout}

\begin{itemize}
\item Specify the data and variables inside the \texttt{ggplot} function.
\item Anything else that goes in here becomes a global setting.
\item Then add layers: geometric objects, statistical models, and facets.
\end{itemize}
\end{frame}

% --------------------------------------------------------------

\begin{frame}[fragile]
\frametitle{Quick note}
\begin{itemize}
\item Never use \texttt{qplot} - short for quick plot.
\item You`ll end up unlearning and relearning a good bit.
\end{itemize}

\end{frame}

% --------------------------------------------------------------

\begin{frame}[fragile]
\frametitle{Let's try an example}
\begin{knitrout}\footnotesize
\definecolor{shadecolor}{rgb}{0.969, 0.969, 0.969}\color{fgcolor}\begin{kframe}
\begin{alltt}
\hlkwd{ggplot}(data = iris, \hlkwd{aes}(x = Sepal.Length, y = Sepal.Width)) +
\hlkwd{geom_point}()
\end{alltt}
\end{kframe}

{\centering \includegraphics[width=.75\linewidth]{figure/first_plot_} 

}



\end{knitrout}

\end{frame}

% --------------------------------------------------------------

\begin{frame}[fragile]
\frametitle{Changing the aesthetics of a geom: \\Increase the size of points}
\begin{knitrout}\footnotesize
\definecolor{shadecolor}{rgb}{0.969, 0.969, 0.969}\color{fgcolor}\begin{kframe}
\begin{alltt}
\hlkwd{ggplot}(data = iris, \hlkwd{aes}(x = Sepal.Length, y = Sepal.Width)) +
\hlkwd{geom_point}(size = 3)
\end{alltt}
\end{kframe}

{\centering \includegraphics[width=.75\linewidth]{figure/first_plot_size_} 

}



\end{knitrout}

\end{frame}

% --------------------------------------------------------------

\begin{frame}[fragile]
\frametitle{Changing the aesthetics of a geom: \\Add some color}
\begin{knitrout}\footnotesize
\definecolor{shadecolor}{rgb}{0.969, 0.969, 0.969}\color{fgcolor}\begin{kframe}
\begin{alltt}
\hlkwd{ggplot}(iris, \hlkwd{aes}(Sepal.Length, Sepal.Width, color = Species)) +
\hlkwd{geom_point}(size = 3)
\end{alltt}
\end{kframe}

{\centering \includegraphics[width=.75\linewidth]{figure/first_plot_color_} 

}



\end{knitrout}

\end{frame}

% --------------------------------------------------------------

\begin{frame}[fragile]
\frametitle{Changing the aesthetics of a geom: \\Differentiate points by shape}
\begin{knitrout}\footnotesize
\definecolor{shadecolor}{rgb}{0.969, 0.969, 0.969}\color{fgcolor}\begin{kframe}
\begin{alltt}
\hlkwd{ggplot}(iris, \hlkwd{aes}(Sepal.Length, Sepal.Width, color = Species)) +
\hlkwd{geom_point}(\hlkwd{aes}(shape = Species), size = 3)
\hlcom{# Why aes(shape = Species)?}
\end{alltt}
\end{kframe}

{\centering \includegraphics[width=.75\linewidth]{figure/first_plot_shape_} 

}



\end{knitrout}

\end{frame}

% --------------------------------------------------------------

\begin{frame}[fragile]
\frametitle{Exercise 1}
\begin{knitrout}\footnotesize
\definecolor{shadecolor}{rgb}{0.969, 0.969, 0.969}\color{fgcolor}\begin{kframe}
\begin{alltt}
\hlcom{# Make a small sample of the diamonds dataset}
d2 <- diamonds[\hlkwd{sample}(1:\hlkwd{dim}(diamonds)[1], 1000), ]
\end{alltt}
\end{kframe}
\end{knitrout}

Then generate this plot below.

\begin{knitrout}\footnotesize
\definecolor{shadecolor}{rgb}{0.969, 0.969, 0.969}\color{fgcolor}

{\centering \includegraphics[width=.75\linewidth]{figure/ex1} 

}



\end{knitrout}

\end{frame}

% --------------------------------------------------------------
% --------------------------------------------------------------

\section*{Stats}
\frame{\sectionpage}

% --------------------------------------------------------------
% --------------------------------------------------------------

\begin{frame}[fragile]
\frametitle{Some terminology}
\begin{columns}[t]

\begin{column}[T]{3cm}
\begin{itemize}
    \item \textbf{\color{gray}data}
    \item \textbf{\color{gray}aesthetics}
    \item \textbf{\color{gray}geometry}
    \item \textbf{stat}s
\end{itemize}
\end{column}

\begin{column}[T]{8cm}
\begin{itemize}
    \item \textbf{Statistical transformations and data summary}
    \item All geoms have associated default stats, and vice versa
    \item e.g. binning for a histogram or fitting a linear model
\end{itemize}
\end{column}

\end{columns}
\end{frame}

% --------------------------------------------------------------

\begin{frame}[fragile]
\frametitle{Built-in stat example: Boxplots}
See \texttt{?geom\_boxplot} for list of options
\begin{knitrout}\footnotesize
\definecolor{shadecolor}{rgb}{0.969, 0.969, 0.969}\color{fgcolor}\begin{kframe}
\begin{alltt}
\hlkwd{library}(MASS)
\hlkwd{ggplot}(birthwt, \hlkwd{aes}(\hlkwd{factor}(race), bwt)) + \hlkwd{geom_boxplot}()
\end{alltt}
\end{kframe}

{\centering \includegraphics[width=.75\linewidth]{figure/boxplots1_} 

}



\end{knitrout}

\end{frame}

% --------------------------------------------------------------

\begin{frame}[fragile]
\frametitle{Built-in stat example: Boxplots}
\begin{knitrout}\footnotesize
\definecolor{shadecolor}{rgb}{0.969, 0.969, 0.969}\color{fgcolor}\begin{kframe}
\begin{alltt}
myplot <- \hlkwd{ggplot}(birthwt, \hlkwd{aes}(\hlkwd{factor}(race), bwt)) + \hlkwd{geom_boxplot}()
\hlkwd{summary}(myplot)
\end{alltt}
\begin{verbatim}
## data: low, age, lwt, race, smoke, ptl, ht, ui, ftv,
##   bwt [189x10]
## mapping:  x = factor(race), y = bwt
## faceting: facet_null() 
## -----------------------------------
## geom_boxplot: outlier.colour = black, outlier.shape = 16, outlier.size = 2, notch = FALSE, notchwidth = 0.5 
## stat_boxplot:  
## position_dodge: (width = NULL, height = NULL)
\end{verbatim}
\end{kframe}
\end{knitrout}

\end{frame}

% --------------------------------------------------------------
% --------------------------------------------------------------

\section*{Facets}
\frame{\sectionpage}

% --------------------------------------------------------------
% --------------------------------------------------------------


\begin{frame}[fragile]
\frametitle{Some terminology}
\begin{columns}[t]

\begin{column}[T]{3cm}
\begin{itemize}
    \item \textbf{\color{gray}data}
    \item \textbf{\color{gray}aesthetics}
    \item \textbf{\color{gray}geometry}
    \item \textbf{\color{gray}stats}
    \item \textbf{facet}s
\end{itemize}
\end{column}

\begin{column}[T]{8cm}
\begin{itemize}
    \item \textbf{Subsetting data to make lattice plots}
    \item Really powerful
\end{itemize}
\end{column}

\end{columns}
\end{frame}

% --------------------------------------------------------------

\begin{frame}[fragile]
\frametitle{Faceting along columns}
\begin{knitrout}\footnotesize
\definecolor{shadecolor}{rgb}{0.969, 0.969, 0.969}\color{fgcolor}\begin{kframe}
\begin{alltt}
\hlkwd{ggplot}(iris, \hlkwd{aes}(Sepal.Length, Sepal.Width, color = Species)) +
\hlkwd{geom_point}() +
\hlkwd{facet_grid}(Species ~ .)
\end{alltt}
\end{kframe}

{\centering \includegraphics[width=.75\linewidth]{figure/facetgrid1} 

}



\end{knitrout}

\end{frame}

% --------------------------------------------------------------

\begin{frame}[fragile]
\frametitle{Faceting along rows}
\begin{knitrout}\footnotesize
\definecolor{shadecolor}{rgb}{0.969, 0.969, 0.969}\color{fgcolor}\begin{kframe}
\begin{alltt}
\hlkwd{ggplot}(iris, \hlkwd{aes}(Sepal.Length, Sepal.Width, color = Species)) +
\hlkwd{geom_point}() +
\hlkwd{facet_grid}(. ~ Species)
\end{alltt}
\end{kframe}

{\centering \includegraphics[width=.75\linewidth]{figure/facet_grid2} 

}



\end{knitrout}

\end{frame}

% --------------------------------------------------------------

\begin{frame}[fragile]
\frametitle{or just wrap your facets}
\begin{knitrout}\footnotesize
\definecolor{shadecolor}{rgb}{0.969, 0.969, 0.969}\color{fgcolor}\begin{kframe}
\begin{alltt}
\hlkwd{ggplot}(iris, \hlkwd{aes}(Sepal.Length, Sepal.Width, color = Species)) +
\hlkwd{geom_point}() +
\hlkwd{facet_wrap}( ~ Species)
\end{alltt}
\end{kframe}

{\centering \includegraphics[width=.75\linewidth]{figure/facet_wrap} 

}



\end{knitrout}

\end{frame}

% --------------------------------------------------------------
% --------------------------------------------------------------

\section*{Scales}
\frame{\sectionpage}

% --------------------------------------------------------------
% --------------------------------------------------------------


\begin{frame}[fragile]
\frametitle{Some terminology}
\begin{columns}[t]

\begin{column}[T]{3cm}
\begin{itemize}
    \item \textbf{\color{gray}data}
    \item \textbf{\color{gray}aesthetics}
    \item \textbf{\color{gray}geometry}
    \item \textbf{\color{gray}stats}
    \item \textbf{\color{gray}facets}
    \item \textbf{scale}s
\end{itemize}
\end{column}

\begin{column}[T]{8cm}
\begin{itemize}
    \item Most often used for adjusting color mapping
\end{itemize}
\end{column}

\end{columns}
\end{frame}

% --------------------------------------------------------------

\begin{frame}[fragile]
\frametitle{Colors}
\begin{knitrout}\footnotesize
\definecolor{shadecolor}{rgb}{0.969, 0.969, 0.969}\color{fgcolor}\begin{kframe}
\begin{alltt}
\hlkwd{aes}(color = variable)  \hlcom{# mapping}
color = \hlstr{"black"}  # setting

\hlcom{# Or add it as a scale}
\hlkwd{scale_fill_manual}(values = \hlkwd{c}(\hlstr{"color1"}, \hlstr{"color2"}))
\end{alltt}
\end{kframe}
\end{knitrout}

\end{frame}

% --------------------------------------------------------------

\begin{frame}[fragile]
\frametitle{The RColorBrewer package}
\begin{knitrout}\footnotesize
\definecolor{shadecolor}{rgb}{0.969, 0.969, 0.969}\color{fgcolor}\begin{kframe}
\begin{alltt}
\hlkwd{library}(RColorBrewer)
\hlkwd{display.brewer.all}()
\end{alltt}
\end{kframe}
\end{knitrout}

\begin{center}
\includegraphics[scale=0.25]{images/color_palette.png}
\end{center}
\end{frame}

% --------------------------------------------------------------

\begin{frame}[fragile]
\frametitle{Using a color brewer palette}
\begin{knitrout}\footnotesize
\definecolor{shadecolor}{rgb}{0.969, 0.969, 0.969}\color{fgcolor}\begin{kframe}
\begin{alltt}
df  <- \hlkwd{melt}(iris, id.vars = \hlstr{"Species"})
\hlkwd{ggplot}(df, \hlkwd{aes}(Species, value, fill = variable)) +
\hlkwd{geom_bar}(stat = \hlstr{"identity"}, position = \hlstr{"dodge"}) +
\hlkwd{scale_fill_brewer}(palette = \hlstr{"Set1"})
\end{alltt}
\end{kframe}

{\centering \includegraphics[width=.75\linewidth]{figure/barcolors} 

}



\end{knitrout}

\end{frame}

% --------------------------------------------------------------

\begin{frame}[fragile]
\frametitle{Manual color scale}
\begin{knitrout}\footnotesize
\definecolor{shadecolor}{rgb}{0.969, 0.969, 0.969}\color{fgcolor}\begin{kframe}
\begin{alltt}
\hlkwd{ggplot}(iris, \hlkwd{aes}(Sepal.Length, Sepal.Width, color = Species)) +
\hlkwd{geom_point}() +
\hlkwd{facet_grid}(Species ~ .) +
\hlkwd{scale_color_manual}(values = \hlkwd{c}(\hlstr{"red"}, \hlstr{"green"}, \hlstr{"blue"}))
\end{alltt}
\end{kframe}

{\centering \includegraphics[width=.75\linewidth]{figure/facetgridcolors} 

}



\end{knitrout}

\end{frame}

% --------------------------------------------------------------

\begin{frame}[fragile]
\frametitle{Adding a continuous scale}
\begin{knitrout}\footnotesize
\definecolor{shadecolor}{rgb}{0.969, 0.969, 0.969}\color{fgcolor}\begin{kframe}
\begin{alltt}
\hlkwd{library}(MASS)
\hlkwd{ggplot}(birthwt, \hlkwd{aes}(\hlkwd{factor}(race), bwt)) +
\hlkwd{geom_boxplot}(width = .2) +
\hlkwd{scale_y_continuous}(labels = (\hlkwd{paste0}(1:4, \hlstr{" Kg"})),
breaks = \hlkwd{seq}(1000, 4000, by = 1000))
\end{alltt}
\end{kframe}

{\centering \includegraphics[width=.75\linewidth]{figure/boxplots3_} 

}



\end{knitrout}

\end{frame}


% --------------------------------------------------------------

\begin{frame}[fragile]
\frametitle{Another continuous scale with custom labels}
\begin{knitrout}\footnotesize
\definecolor{shadecolor}{rgb}{0.969, 0.969, 0.969}\color{fgcolor}\begin{kframe}
\begin{alltt}
\hlcom{# Assign the plot to an object}
dd <- \hlkwd{ggplot}(iris, \hlkwd{aes}(Sepal.Length, Sepal.Width, color = Species)) +
\hlkwd{geom_point}(size = 4, shape = 16) +
\hlkwd{facet_grid}(. ~Species)
\hlcom{# Now add a scale}
dd +
\hlkwd{scale_y_continuous}(breaks = \hlkwd{seq}(2, 8, by = 1),
labels = \hlkwd{paste0}(2:8, \hlstr{" cm"}))
\end{alltt}
\end{kframe}
\end{knitrout}

\end{frame}

% --------------------------------------------------------------

\begin{frame}[fragile]
\frametitle{gradients}
\begin{knitrout}\footnotesize
\definecolor{shadecolor}{rgb}{0.969, 0.969, 0.969}\color{fgcolor}\begin{kframe}
\begin{alltt}
h + \hlkwd{geom_histogram}( \hlkwd{aes}(fill = ..count..), color=\hlstr{"black"}) +
\hlkwd{scale_fill_gradient}(low=\hlstr{"green"}, high=\hlstr{"red"})
\end{alltt}


{\ttfamily\noindent\bfseries\color{errorcolor}{\#\# Error: object 'h' not found}}\end{kframe}
\end{knitrout}

\end{frame}

% --------------------------------------------------------------

\begin{frame}[fragile]
\frametitle{Refer to a color chart for beautful visualizations}
\begin{center}
\url{http://tools.medialab.sciences-po.fr/iwanthue/}
\newline
\newline
\includegraphics[scale=0.25]{images/color_schemes.png}
\end{center}
\end{frame}

% --------------------------------------------------------------

\begin{frame}[fragile]
\frametitle{Commonly used scales}
\begin{knitrout}\footnotesize
\definecolor{shadecolor}{rgb}{0.969, 0.969, 0.969}\color{fgcolor}\begin{kframe}
\begin{alltt}
\hlkwd{scale_fill_discrete}(); \hlkwd{scale_colour_discrete}()
\hlkwd{scale_fill_hue}(); \hlkwd{scale_color_hue}()
\hlkwd{scale_fill_manual}();  \hlkwd{scale_color_manual}()
\hlkwd{scale_fill_brewer}(); \hlkwd{scale_color_brewer}()
\hlkwd{scale_linetype}(); \hlkwd{scale_shape_manual}()
\end{alltt}
\end{kframe}
\end{knitrout}

\end{frame}

% --------------------------------------------------------------
% --------------------------------------------------------------

\section*{Coordinates}
\frame{\sectionpage}

% --------------------------------------------------------------
% --------------------------------------------------------------

\begin{frame}[fragile]
\frametitle{Some terminology}
\begin{columns}[t]

\begin{column}[T]{3cm}
\begin{itemize}
    \item \textbf{\color{gray}data}
    \item \textbf{\color{gray}aesthetics}
    \item \textbf{\color{gray}geometry}
    \item \textbf{\color{gray}stats}
    \item \textbf{\color{gray}facets}
    \item \textbf{\color{gray}scales}
    \item \textbf{coord}inates
\end{itemize}
\end{column}

\begin{column}[T]{8cm}
\begin{itemize}
    \item Not going to cover this detail
    \item for e.g. polar coordinate plots
\end{itemize}
\end{column}

\end{columns}
\end{frame}

% --------------------------------------------------------------
% --------------------------------------------------------------

\section*{Putting it all together with more examples}
\frame{\sectionpage}

% --------------------------------------------------------------
% --------------------------------------------------------------

\section*{Histograms}
\frame{\sectionpage}

% --------------------------------------------------------------
% --------------------------------------------------------------

\begin{frame}[fragile]
See \texttt{?geom\_histogram} for list of options
\begin{knitrout}\footnotesize
\definecolor{shadecolor}{rgb}{0.969, 0.969, 0.969}\color{fgcolor}\begin{kframe}
\begin{alltt}
h <- \hlkwd{ggplot}(faithful, \hlkwd{aes}(x = waiting))
h + \hlkwd{geom_histogram}(binwidth = 30, colour = \hlstr{"black"})
\end{alltt}
\end{kframe}

{\centering \includegraphics[width=.75\linewidth]{figure/histogr_} 

}



\end{knitrout}

\end{frame}

% --------------------------------------------------------------

\begin{frame}[fragile]
\begin{knitrout}\footnotesize
\definecolor{shadecolor}{rgb}{0.969, 0.969, 0.969}\color{fgcolor}\begin{kframe}
\begin{alltt}
h <- \hlkwd{ggplot}(faithful, \hlkwd{aes}(x = waiting))
h + \hlkwd{geom_histogram}(binwidth = 8, fill = \hlstr{"steelblue"},
colour = \hlstr{"black"})
\end{alltt}
\end{kframe}

{\centering \includegraphics[width=.75\linewidth]{figure/histogra_} 

}



\end{knitrout}

\end{frame}

% --------------------------------------------------------------
% --------------------------------------------------------------

\section*{Line plots}
\frame{\sectionpage}

% --------------------------------------------------------------
% --------------------------------------------------------------

\begin{frame}[fragile]


% climate <- read.csv(text = RCurl::getURL('https://raw.github.com/karthikram/ggplot-lecture/master/climate.csv'))
\begin{knitrout}\footnotesize
\definecolor{shadecolor}{rgb}{0.969, 0.969, 0.969}\color{fgcolor}\begin{kframe}
\begin{alltt}
climate <- \hlkwd{read.csv}(\hlstr{"../data/climate.csv"}, header = T)
\hlkwd{ggplot}(climate, \hlkwd{aes}(Year, Anomaly10y)) +
\hlkwd{geom_line}()
\end{alltt}
\end{kframe}

{\centering \includegraphics[width=.75\linewidth]{figure/linea_} 

}



\end{knitrout}

\begin{flushright}
\begingroup
    \fontsize{6pt}{12pt}\selectfont
    \begin{verbatim}
        climate <- read.csv(text =
        RCurl::getURL('https://raw.github.com/karthikram/ggplot-lecture/master/climate.csv'))
    \end{verbatim}
\endgroup
\end{flushright}
\end{frame}

% --------------------------------------------------------------

\begin{frame}[fragile]
We can also plot confidence regions
\begin{knitrout}\footnotesize
\definecolor{shadecolor}{rgb}{0.969, 0.969, 0.969}\color{fgcolor}\begin{kframe}
\begin{alltt}
\hlkwd{ggplot}(climate, \hlkwd{aes}(Year, Anomaly10y)) +
\hlkwd{geom_ribbon}(\hlkwd{aes}(ymin = Anomaly10y - Unc10y,
ymax = Anomaly10y + Unc10y),
fill = \hlstr{"blue"}, alpha = .1) +
\hlkwd{geom_line}(color = \hlstr{"steelblue"})
\end{alltt}
\end{kframe}

{\centering \includegraphics[width=.75\linewidth]{figure/lineb_} 

}



\end{knitrout}

\end{frame}

% --------------------------------------------------------------

\begin{frame}[fragile]
\frametitle{Exercise 2}
\begin{itemize}
\item Modify the previous plot and change it such that there are three lines instead of one with a confidence band.
\begin{knitrout}\footnotesize
\definecolor{shadecolor}{rgb}{0.969, 0.969, 0.969}\color{fgcolor}

{\centering \includegraphics[width=.75\linewidth]{figure/ex2} 

}



\end{knitrout}


\end{itemize}
\end{frame}

% --------------------------------------------------------------
% --------------------------------------------------------------

\section*{Bar plots}
\frame{\sectionpage}

% --------------------------------------------------------------
% --------------------------------------------------------------

\begin{frame}[fragile]
\begin{knitrout}\footnotesize
\definecolor{shadecolor}{rgb}{0.969, 0.969, 0.969}\color{fgcolor}\begin{kframe}
\begin{alltt}
\hlkwd{ggplot}(iris, \hlkwd{aes}(Species, Sepal.Length)) +
\hlkwd{geom_bar}(stat = \hlstr{"identity"})
\end{alltt}
\end{kframe}

{\centering \includegraphics[width=.75\linewidth]{figure/barone_} 

}



\end{knitrout}

\end{frame}

% --------------------------------------------------------------

\begin{frame}[fragile]
\begin{knitrout}\footnotesize
\definecolor{shadecolor}{rgb}{0.969, 0.969, 0.969}\color{fgcolor}\begin{kframe}
\begin{alltt}
df  <- \hlkwd{melt}(iris, id.vars = \hlstr{"Species"})
\hlkwd{ggplot}(df, \hlkwd{aes}(Species, value, fill = variable)) +
\hlkwd{geom_bar}(stat = \hlstr{"identity"})
\end{alltt}
\end{kframe}

{\centering \includegraphics[width=.75\linewidth]{figure/bartwo_} 

}



\end{knitrout}

\end{frame}

% --------------------------------------------------------------
% --------------------------------------------------------------

\section*{Density Plots}
\frame{\sectionpage}

% --------------------------------------------------------------
% --------------------------------------------------------------

\begin{frame}[fragile]
\frametitle{Density plots}
\begin{knitrout}\footnotesize
\definecolor{shadecolor}{rgb}{0.969, 0.969, 0.969}\color{fgcolor}\begin{kframe}
\begin{alltt}
\hlkwd{ggplot}(faithful, \hlkwd{aes}(waiting)) + \hlkwd{geom_density}()
\end{alltt}
\end{kframe}

{\centering \includegraphics[width=.75\linewidth]{figure/densityone_} 

}



\end{knitrout}

\end{frame}

% --------------------------------------------------------------

\begin{frame}[fragile]
\frametitle{Density plots}
\begin{knitrout}\footnotesize
\definecolor{shadecolor}{rgb}{0.969, 0.969, 0.969}\color{fgcolor}\begin{kframe}
\begin{alltt}
\hlkwd{ggplot}(faithful, \hlkwd{aes}(waiting)) +
\hlkwd{geom_density}(fill = \hlstr{"blue"}, alpha = 0.1)
\end{alltt}
\end{kframe}

{\centering \includegraphics[width=.75\linewidth]{figure/densityonefove_} 

}



\end{knitrout}

\end{frame}

% --------------------------------------------------------------

\begin{frame}[fragile]
\begin{knitrout}\footnotesize
\definecolor{shadecolor}{rgb}{0.969, 0.969, 0.969}\color{fgcolor}\begin{kframe}
\begin{alltt}
\hlkwd{ggplot}(faithful, \hlkwd{aes}(waiting)) +
\hlkwd{geom_line}(stat = \hlstr{"density"})
\end{alltt}
\end{kframe}

{\centering \includegraphics[width=.75\linewidth]{figure/densitytwo___} 

}



\end{knitrout}

\end{frame}

% --------------------------------------------------------------
% --------------------------------------------------------------

\section*{Adding smoothers}
\frame{\sectionpage}

% --------------------------------------------------------------
% --------------------------------------------------------------

\begin{frame}[fragile]
\begin{knitrout}\footnotesize
\definecolor{shadecolor}{rgb}{0.969, 0.969, 0.969}\color{fgcolor}\begin{kframe}
\begin{alltt}
\hlkwd{ggplot}(iris, \hlkwd{aes}(Sepal.Length, Sepal.Width, color = Species)) +
\hlkwd{geom_point}(\hlkwd{aes}(shape = Species), size = 3) +
\hlkwd{geom_smooth}(method = \hlstr{"lm"})
\end{alltt}
\end{kframe}

{\centering \includegraphics[width=.75\linewidth]{figure/adding_stats_} 

}



\end{knitrout}

\end{frame}

% --------------------------------------------------------------

\begin{frame}[fragile]
\begin{knitrout}\footnotesize
\definecolor{shadecolor}{rgb}{0.969, 0.969, 0.969}\color{fgcolor}\begin{kframe}
\begin{alltt}
\hlkwd{ggplot}(iris, \hlkwd{aes}(Sepal.Length, Sepal.Width, color = Species)) +
\hlkwd{geom_point}(\hlkwd{aes}(shape = Species), size = 3) +
\hlkwd{geom_smooth}(method = \hlstr{"lm"}) +
\hlkwd{facet_grid}(. ~ Species)
\end{alltt}
\end{kframe}

{\centering \includegraphics[width=.75\linewidth]{figure/adding_stats2_} 

}



\end{knitrout}

\end{frame}

% --------------------------------------------------------------
% --------------------------------------------------------------

\section*{\texttt{plyr} and \texttt{reshape} are key for using \texttt{R}}
\frame{\sectionpage}

% --------------------------------------------------------------
% --------------------------------------------------------------

\begin{frame}[fragile]
    \frametitle{\texttt{plyr} and \texttt{reshape}}
    These two packages are the swiss army knives of \texttt{R}.
\begin{itemize}
\item \texttt{plyr}
    \begin{enumerate}
    \item ddply
    \item llply
    \item join
    \end{enumerate}
\item \texttt{reshape}.
    \begin{enumerate}
    \item melt
    \item dcast
    \item acast
    \end{enumerate}
\end{itemize}
\end{frame}

% --------------------------------------------------------------

\begin{frame}[fragile]
\begin{knitrout}\footnotesize
\definecolor{shadecolor}{rgb}{0.969, 0.969, 0.969}\color{fgcolor}\begin{kframe}
\begin{alltt}
iris[1:2, ]
\end{alltt}
\begin{verbatim}
##   Sepal.Length Sepal.Width Petal.Length Petal.Width Species
## 1          5.1         3.5          1.4         0.2  setosa
## 2          4.9         3.0          1.4         0.2  setosa
\end{verbatim}
\begin{alltt}
df  <- \hlkwd{melt}(iris, id.vars = \hlstr{"Species"})
df[1:2, ]
\end{alltt}
\begin{verbatim}
##   Species     variable value
## 1  setosa Sepal.Length   5.1
## 2  setosa Sepal.Length   4.9
\end{verbatim}
\end{kframe}
\end{knitrout}

\end{frame}

% --------------------------------------------------------------

\begin{frame}[fragile]
\begin{knitrout}\footnotesize
\definecolor{shadecolor}{rgb}{0.969, 0.969, 0.969}\color{fgcolor}\begin{kframe}
\begin{alltt}
\hlkwd{ggplot}(df, \hlkwd{aes}(Species, value, fill = variable)) +
\hlkwd{geom_bar}(stat = \hlstr{"identity"}, position = \hlstr{"dodge"})
\end{alltt}
\end{kframe}

{\centering \includegraphics[width=.75\linewidth]{figure/barthree_} 

}



\end{knitrout}

\end{frame}

% --------------------------------------------------------------

\begin{frame}[fragile]
\frametitle{Exercise 3}
Using the d2 dataset you created earlier, generate this plot below. Take a quick look at the data first to see if it needs to be binned.
\begin{knitrout}\footnotesize
\definecolor{shadecolor}{rgb}{0.969, 0.969, 0.969}\color{fgcolor}

{\centering \includegraphics[width=.75\linewidth]{figure/ex3} 

}



\end{knitrout}

\end{frame}

% --------------------------------------------------------------

\begin{frame}[fragile]
\frametitle{Exercise 4}
\begin{itemize}
\item Using the climate dataset, create a new variable called sign. Make it logical (true/false) based on the sign of Anomaly10y.
\item Plot a bar plot and use \texttt{sign} variable as the fill.\\
\begin{knitrout}\footnotesize
\definecolor{shadecolor}{rgb}{0.969, 0.969, 0.969}\color{fgcolor}\begin{kframe}


{\ttfamily\noindent\bfseries\color{errorcolor}{\#\# Error: cannot open the connection}}

{\ttfamily\noindent\bfseries\color{errorcolor}{\#\# Error: object 'clim' not found}}

{\ttfamily\noindent\bfseries\color{errorcolor}{\#\# Error: object 'clim' not found}}\end{kframe}
\end{knitrout}


\end{itemize}
\end{frame}

% --------------------------------------------------------------
% --------------------------------------------------------------


\section*{Themes}
\frame{\sectionpage}

% --------------------------------------------------------------
% --------------------------------------------------------------

\begin{frame}[fragile]
\frametitle{Adding themes}
Themes are a great way to define custom plots.
\begin{knitrout}\footnotesize
\definecolor{shadecolor}{rgb}{0.969, 0.969, 0.969}\color{fgcolor}\begin{kframe}
\begin{alltt}
+\hlkwd{theme}()
\hlcom{# see ?theme() for more options}
\end{alltt}
\end{kframe}
\end{knitrout}

\end{frame}

% --------------------------------------------------------------

\begin{frame}[fragile]
\frametitle{A themed plot}
\begin{knitrout}\footnotesize
\definecolor{shadecolor}{rgb}{0.969, 0.969, 0.969}\color{fgcolor}\begin{kframe}
\begin{alltt}
\hlkwd{ggplot}(iris, \hlkwd{aes}(Sepal.Length, Sepal.Width, color = Species)) +
\hlkwd{geom_point}(size = 1.2, shape = 16) +
\hlkwd{facet_wrap}( ~ Species) +
\hlkwd{theme}(legend.key = \hlkwd{element_rect}(fill = NA),
legend.position = \hlstr{"bottom"},
strip.background = \hlkwd{element_rect}(fill = NA),
axis.title.y = \hlkwd{element_text}(angle = 0))
\end{alltt}
\end{kframe}
\end{knitrout}

\end{frame}

% --------------------------------------------------------------

\begin{frame}[fragile]
\frametitle{Adding themes}
\begin{knitrout}\footnotesize
\definecolor{shadecolor}{rgb}{0.969, 0.969, 0.969}\color{fgcolor}
\includegraphics[width=.75\linewidth]{figure/facet_wrap_theme_execc} 

\end{knitrout}

\end{frame}

% --------------------------------------------------------------

\begin{frame}[fragile]
\frametitle{ggthemes library}
\begin{knitrout}\footnotesize
\definecolor{shadecolor}{rgb}{0.969, 0.969, 0.969}\color{fgcolor}\begin{kframe}
\begin{alltt}
\hlkwd{install.packages}(\hlstr{"ggthemes"})
\hlkwd{library}(ggthemes)
\hlcom{# Then add one of these themes to your plot}
+\hlkwd{theme_stata}()
+\hlkwd{theme_excel}()
+\hlkwd{theme_wsj}()
+\hlkwd{theme_solarized}()
\end{alltt}
\end{kframe}
\end{knitrout}

\end{frame}

% --------------------------------------------------------------
% --------------------------------------------------------------

\section*{Create functions to automate your plotting}
\frame{\sectionpage}

% --------------------------------------------------------------
% --------------------------------------------------------------

\begin{frame}[fragile]
\frametitle{Write functions for day to day plots}
\begin{knitrout}\footnotesize
\definecolor{shadecolor}{rgb}{0.969, 0.969, 0.969}\color{fgcolor}\begin{kframe}
\begin{alltt}
my_custom_plot <- \hlkwd{function}(df, title = \hlstr{""}, ...) \{
    \hlkwd{ggplot}(df, ...) +
    \hlkwd{ggtitle}(title) +
    \hlkwd{whatever_geoms}() +
    \hlkwd{theme}(...)
\}
\end{alltt}
\end{kframe}
\end{knitrout}


Then just call your function to generate a plot.
It's a lot easier to fix one function that do it over and over for many plots
\begin{knitrout}\footnotesize
\definecolor{shadecolor}{rgb}{0.969, 0.969, 0.969}\color{fgcolor}\begin{kframe}
\begin{alltt}
plot1 <- \hlkwd{my_custom_plot}(dataset1, title = \hlstr{"Figure 1"})
\end{alltt}
\end{kframe}
\end{knitrout}


\end{frame}

% --------------------------------------------------------------
% --------------------------------------------------------------


\section*{Publication quality figures}
\frame{\sectionpage}

% --------------------------------------------------------------
% --------------------------------------------------------------

% How to save your plots

\begin{frame}[fragile]
\begin{itemize}
\item If the plot is on your screen
\begin{knitrout}\footnotesize
\definecolor{shadecolor}{rgb}{0.969, 0.969, 0.969}\color{fgcolor}\begin{kframe}
\begin{alltt}
\hlkwd{ggsave}(\hlstr{"~/path/to/figure/filename.png"})
\end{alltt}
\end{kframe}
\end{knitrout}

\item If your plot is assigned to an object
\begin{knitrout}\footnotesize
\definecolor{shadecolor}{rgb}{0.969, 0.969, 0.969}\color{fgcolor}\begin{kframe}
\begin{alltt}
\hlkwd{ggsave}(plot1, file = \hlstr{"~/path/to/figure/filename.png"})
\end{alltt}
\end{kframe}
\end{knitrout}


\item Specify a size
\begin{knitrout}\footnotesize
\definecolor{shadecolor}{rgb}{0.969, 0.969, 0.969}\color{fgcolor}\begin{kframe}
\begin{alltt}
\hlkwd{ggsave}(file = \hlstr{"/path/to/figure/filename.png"}, width = 6,
height =4)
\end{alltt}
\end{kframe}
\end{knitrout}

\item or any format (pdf, png, eps, svg, jpg)
\begin{knitrout}\footnotesize
\definecolor{shadecolor}{rgb}{0.969, 0.969, 0.969}\color{fgcolor}\begin{kframe}
\begin{alltt}
\hlkwd{ggsave}(file = \hlstr{"/path/to/figure/filename.eps"})
\hlkwd{ggsave}(file = \hlstr{"/path/to/figure/filename.jpg"})
\hlkwd{ggsave}(file = \hlstr{"/path/to/figure/filename.pdf"})
\end{alltt}
\end{kframe}
\end{knitrout}

\end{itemize}
\end{frame}

% --------------------------------------------------------------

\begin{frame}[fragile]
\frametitle{Further help}
\begin{itemize}
\item You've just scratched the surface with ggplot2.
\item Practice
\item Read the docs (either locally in \texttt{R} or at \url{http://docs.ggplot2.org/current/})
\item Work together
\end{itemize}
\begin{center}
\includegraphics[scale=.15]{images/chang_book.png}
\includegraphics[scale=.15]{images/hadley.png}
\end{center}
\end{frame}

% --------------------------------------------------------------
% end, hope it was useful.
\end{document}
